% !TEX program = pdflatex
\documentclass[11pt,a4paper]{article}

\usepackage[margin=2.2cm]{geometry}
\usepackage[T2A]{fontenc}
\usepackage[utf8]{inputenc}
\usepackage[russian]{babel}
\usepackage{amsmath,amssymb,amsthm}
\usepackage{mathtools}
\usepackage{microtype}
\usepackage{hyperref}

\hypersetup{
  colorlinks=true,
  linkcolor=blue,
  urlcolor=blue,
  citecolor=blue
}

\begin{document}

\section*{Задача 2}
Рассмотрим блуждание на неориентированном графе. Частица (цепь Маркова) начинает в вершине $1$ и далее на каждом шаге:
\begin{itemize}
  \item с вероятностью $\alpha>0$ остаётся в текущей вершине;
  \item с вероятностью $1-\alpha$ переходит равновероятно в одну из соседних вершин.
\end{itemize}

\begin{enumerate}
  \item Граф --- путь $1-2-\dots-k$. Описать матрицу переходов, найти инвариантные распределения, доказать эргодичность.
  \item Граф --- два несвязанных пути $1-2-\dots-k$ и $(k+1)-\dots-(k+m)$. Описать матрицу переходов, найти инвариантные распределения, проверить эргодичность.
  \item Граф --- цикл $1-2-\dots-k-1$. Найти инвариантное распределение и доказать эргодичность.
\end{enumerate}

\section*{Обозначим пару частых фактов, на которые буду ссылаться}

\begin{itemize}
\item \textbf{Факт A (про стационарное распределение для неориентированного графа).}
Для (ленивого или неленивого) случайного блуждания на конечном неориентированном графе стационарное распределение
пропорционально степеням: $\pi(v)\propto \deg(v)$.

\item \textbf{Факт B (про эргодичность).}
Конечная цепь Маркова эргодична $\Leftrightarrow$ она неприводима и апериодична.
Здесь: неприводимость эквивалентна связности графа (по рёбрам можно добраться в любую вершину),
а апериодичность при $\alpha>0$ получается сразу, потому что есть самопереход $P_{ii}=\alpha>0$ (значит период $1$).
\end{itemize}

\section*{1) Путь $1-2-\dots-k$}

\subsection*{Матрица переходов}
Состояния: $S=\{1,2,\dots,k\}$.

Степени вершин: $\deg(1)=\deg(k)=1$, а для $2\le i\le k-1$ имеем $\deg(i)=2$.

Переходы:
\[
P_{ii}=\alpha\quad \text{для всех } i.
\]
Для концов:
\[
P_{1,2}=1-\alpha,\qquad P_{k,k-1}=1-\alpha,
\]
и других ненулевых переходов из $1$ и $k$ нет.
Для внутренних вершин $2\le i\le k-1$:
\[
P_{i,i-1}=P_{i,i+1}=\frac{1-\alpha}{2}.
\]
Остальные элементы равны $0$.

То есть $P$:
\[
P=
\begin{pmatrix}
\alpha & 1-\alpha & 0 & \cdots & 0\\
\dfrac{1-\alpha}{2} & \alpha & \dfrac{1-\alpha}{2} & \cdots & 0\\
0 & \dfrac{1-\alpha}{2} & \alpha & \ddots & \vdots\\
\vdots & \ddots & \ddots & \ddots & \dfrac{1-\alpha}{2}\\
0 & \cdots & 0 & 1-\alpha & \alpha
\end{pmatrix}.
\]

\subsection*{Инвариантное распределение}
По Факту A стационарное распределение пропорционально степеням:
\[
\pi(i)=\frac{\deg(i)}{\sum_{x=1}^k \deg(x)}.
\]
В пути $|E|=k-1$, а сумма степеней равна $2|E|=2(k-1)$. Значит
\[
\pi(1)=\pi(k)=\frac{1}{2(k-1)},\qquad
\pi(i)=\frac{2}{2(k-1)}=\frac{1}{k-1}\quad (2\le i\le k-1).
\]

\subsection*{Эргодичность}
Путь связен $\Rightarrow$ цепь неприводима (частица может пройти по рёбрам от любой вершины к любой).
Также $\alpha>0 \Rightarrow P_{ii}=\alpha>0$, значит цепь апериодична.
По Факту B цепь эргодична. В частности, стационарное распределение единственно.

\section*{2) Два несвязанных пути $1-2-\dots-k$ и $(k+1)-\dots-(k+m)$}

\subsection*{Матрица переходов}
Состояния: $S=\{1,2,\dots,k+m\}$.

Переходы возможны только внутри своей компоненты связности, поэтому матрица $P$ блочно-диагональная:
\[
P=
\begin{pmatrix}
P^{(1)} & 0\\
0 & P^{(2)}
\end{pmatrix},
\]
где $P^{(1)}$ --- матрица из пункта 1 на вершинах $\{1,\dots,k\}$,
а $P^{(2)}$ --- такая же матрица для пути на вершинах $\{k+1,\dots,k+m\}$
(с теми же формулами, просто индексы сдвинуты).

Иными словами:
\begin{itemize}
\item если $i\in\{1,\dots,k\}$, то $P_{ij}=0$ при $j\notin\{1,\dots,k\}$;
\item если $i\in\{k+1,\dots,k+m\}$, то $P_{ij}=0$ при $j\notin\{k+1,\dots,k+m\}$;
\item внутри каждого блока действуют правила ленивого блуждания по соответствующему пути.
\end{itemize}

\subsection*{Инвариантные распределения}
Здесь две замкнутые компоненты:
\[
C_1=\{1,\dots,k\},\qquad C_2=\{k+1,\dots,k+m\}.
\]
Цепь не может переносить вероятность между $C_1$ и $C_2$ (между блоками нули), поэтому стационарные распределения
не единственны: можно ``положить'' любую долю массы на первую компоненту и оставшуюся на вторую.

Пусть $\pi^{(1)}$ --- стационарное распределение на $C_1$, продолженное нулями на $C_2$:
\[
\pi^{(1)}(i)=
\begin{cases}
\dfrac{1}{2(k-1)}, & i\in\{1,k\},\\[6pt]
\dfrac{1}{k-1}, & i\in\{2,\dots,k-1\},\\[6pt]
0, & i\in C_2.
\end{cases}
\]
Аналогично $\pi^{(2)}$ --- стационарное распределение на $C_2$, продолженное нулями на $C_1$:
\[
\pi^{(2)}(j)=
\begin{cases}
0, & j\in C_1,\\[4pt]
\dfrac{1}{2(m-1)}, & j\in\{k+1,k+m\},\\[6pt]
\dfrac{1}{m-1}, & j\in\{k+2,\dots,k+m-1\}.
\end{cases}
\]

Любое инвариантное распределение имеет вид
\[
\pi=c\,\pi^{(1)}+(1-c)\,\pi^{(2)},\qquad c\in[0,1].
\]
\emph{Пояснение:} $c$ --- это просто общая масса вероятности, которую $\pi$ кладёт на $C_1$,
а внутри каждой компоненты стационарность заставляет распределение быть пропорциональным степеням (Факт A),
как в пункте 1.

\subsection*{Эргодичность}
Её нет. Граф несвязен $\Rightarrow$ цепь приводима $\Rightarrow$ эргодичности нет (Факт B требует неприводимость).

\paragraph{Замечание про старт из вершины 1.}
Если $X_0=1\in C_1$, то с вероятностью $1$ цепь навсегда остаётся в $C_1$,
и внутри $C_1$ она будет сходиться к $\pi^{(1)}$ (но на всём $S$ цепь всё равно не эргодическая).

\section*{3) Цикл $1-2-\dots-k-1$}

\subsection*{Матрица переходов}
Состояния: $S=\{1,2,\dots,k\}$.
У каждой вершины в цикле степень $\deg(i)=2$.
Соседи вершины $i$: $i-1$ и $i+1$ по модулю $k$ (то есть $0\equiv k$, $k+1\equiv 1$).

Переходы:
\[
P_{ii}=\alpha,\qquad
P_{i,i-1}=P_{i,i+1}=\frac{1-\alpha}{2}\quad (\text{индексы по модулю }k),
\]
остальные элементы $0$.

\subsection*{Инвариантное распределение}
По Факту A: $\pi(v)\propto \deg(v)$.
Но степени равны, значит распределение равномерно:
\[
\pi(i)=\frac{1}{k}\quad \text{для всех } i.
\]

\subsection*{Эргодичность}
Цикл связен $\Rightarrow$ цепь неприводима.
$\alpha>0 \Rightarrow$ есть самопереход в каждой вершине $\Rightarrow$ апериодичность.
По Факту B цепь эргодична.

\end{document}
