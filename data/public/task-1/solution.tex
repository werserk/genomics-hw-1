% !TEX program = pdflatex
\documentclass[12pt,a4paper]{article}

\usepackage[margin=2.2cm]{geometry}
\usepackage[T2A]{fontenc}
\usepackage[utf8]{inputenc}
\usepackage[russian,english]{babel}
\usepackage{amsmath,amssymb,amsthm}
\usepackage{mathtools}
\usepackage{microtype}
\usepackage{graphicx}
\usepackage{hyperref}

\hypersetup{
  colorlinks=true,
  linkcolor=blue,
  urlcolor=blue,
  citecolor=blue
}

\begin{document}

\section*{Задача}
На контрольной работе, которую пишут три группы в разных аудиториях, есть по одному постоянному проверяющему и один главный проверяющий (ГП), который периодически перемещается между аудиториями. Каждую минуту ГП может переместиться, либо остаться на месте с вероятностью $0.2$; его решения не зависят от истории предыдущих перемещений. Если он перемещается, то выбирает новую аудиторию случайно и равновероятно из доступных.

Будем моделировать поведение ГП как цепь Маркова.

\begin{enumerate}
  \item Опишите явно, какие у этой цепи состояния и какая матрица перехода.
  \item Если проверяющий стоит в стартовый момент в аудитории $R1$, каковы вероятности его нахождения в разных аудиториях через $10$ минут?
  \item Если контрольная длится $120$ минут и ГП изначально следит в одной из аудиторий, какова вероятность, что на последней минуте он будет в той же аудитории?
\end{enumerate}

\section*{Решение}

\subsection*{Схема}
\begin{center}
  \includegraphics[width=0.8\linewidth]{schema.png}
\end{center}

\subsection*{1) Состояния и матрица переходов}
\paragraph{Состояния.}
Пусть аудитории:
\[
S=\{R1,R2,R3\}.
\]
Определим $X_t\in S$ как аудиторию, в которой находится ГП в минуту $t$.

\paragraph{Переходы за 1 минуту.}
Из условия:
\begin{itemize}
  \item с вероятностью $0.2$ ГП остаётся в текущей аудитории;
  \item с вероятностью $0.8$ он уходит в \emph{другую} аудиторию, выбирая равновероятно одну из двух оставшихся.
\end{itemize}
Поэтому для любого $i$:
\[
\mathbb P(X_{t+1}=Ri\mid X_t=Ri)=0.2,
\]
а для любого $j\neq i$:
\[
\mathbb P(X_{t+1}=Rj\mid X_t=Ri)=\frac{0.8}{2}=0.4.
\]

\paragraph{Матрица перехода $P$.}
В порядке состояний $(R1,R2,R3)$:
\[
P=\begin{pmatrix}
0.2 & 0.4 & 0.4\\
0.4 & 0.2 & 0.4\\
0.4 & 0.4 & 0.2
\end{pmatrix}.
\]

\subsection*{2) Распределение через 10 минут при старте в $R1$}
Стартовое распределение:
\[
\mu_0=(1,0,0).
\]
Через $n$ минут:
\[
\mu_n=\mu_0 P^n.
\]

\paragraph{Удобная формула для $P^n$ (за счёт симметрии).}
Обозначим через $J$ матрицу $3\times 3$ из единиц и введём проектор на равномерное распределение:
\[
\Pi=\frac{1}{3}J.
\]
Тогда
\[
P=-0.2\,I+0.4\,J=\Pi+(-0.2)(I-\Pi).
\]
У матриц $\Pi$ и $(I-\Pi)$ выполняются тождества:
\[
\Pi^2=\Pi,\qquad (I-\Pi)^2=(I-\Pi),\qquad \Pi(I-\Pi)=0.
\]
Отсюда для любого $n\ge 1$:
\[
P^n=\Pi+(-0.2)^n\,(I-\Pi).
\]

Применим это к $\mu_0$:
\[
\mu_n=\mu_0\Pi+(-0.2)^n\bigl(\mu_0-\mu_0\Pi\bigr).
\]
\[
\mu_0\Pi=\left(\frac{1}{3},\frac{1}{3},\frac{1}{3}\right),
\]
\[
\mu_n=\left(\frac{1}{3},\frac{1}{3},\frac{1}{3}\right)+(-0.2)^n\left(\frac{2}{3},-\frac{1}{3},-\frac{1}{3}\right).
\]

Подставим $n=10$
\[
(-0.2)^{10}=0.2^{10}=\left(\frac{1}{5}\right)^{10}=\frac{1}{9\,765\,625}.
\]

Тогда
\[
\begin{aligned}
\mathbb P(X_{10}=R1)&=\frac{1}{3}+\frac{2}{3}\cdot\frac{1}{9\,765\,625}
=\frac{3\,255\,209}{9\,765\,625}\approx 0.3333334016,\\[4pt]
\mathbb P(X_{10}=R2)&=\frac{1}{3}-\frac{1}{3}\cdot\frac{1}{9\,765\,625}
=\frac{3\,255\,208}{9\,765\,625}\approx 0.3333332992,\\[4pt]
\mathbb P(X_{10}=R3)&=\frac{1}{3}-\frac{1}{3}\cdot\frac{1}{9\,765\,625}
=\frac{3\,255\,208}{9\,765\,625}\approx 0.3333332992.
\end{aligned}
\]

\subsection*{3) Вероятность быть в исходной аудитории на последней минуте (120 минут)}
Если стартуем в $R1$, то вероятность вернуться в $R1$ через $n$ минут --- это первая координата $\mu_n$:
\[
\mathbb P(X_n=R1\mid X_0=R1)=\frac{1}{3}+\frac{2}{3}(-0.2)^n.
\]

Для $n=120$:
\[
\mathbb P(X_{120}=R1\mid X_0=R1)=\frac{1}{3}+\frac{2}{3}\cdot 0.2^{120}
=\frac{1}{3}+\frac{2}{3}\cdot 5^{-120}.
\]
Добавка $\frac{2}{3}\cdot 0.2^{120}$ крайне мала, поэтому практически
\[
\mathbb P(X_{120}=\text{стартовая аудитория})\approx \frac{1}{3}.
\]

\end{document}
